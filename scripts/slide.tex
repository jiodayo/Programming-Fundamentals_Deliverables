% !TEX program = xelatex

\documentclass[aspectratio=169, 14pt]{beamer} % aspectratio=43, 149, 169
\usepackage{fontspec}
\usepackage{xeCJK}
\usepackage{here, amsmath, latexsym, amssymb, bm, ascmac, mathtools, multicol, tcolorbox, subfig, url}
\usepackage{graphicx}
\usepackage{pgfpages}
\usepackage{ascmac}
\usepackage{bm}
\usepackage{atbegshi} %しおりの文字化け解消
\usepackage{xcolor}
\usepackage{bookmark}
\usepackage{appendixnumberbeamer}

\newcounter{mainframenumber}
\setcounter{mainframenumber}{0}

\usetheme{metropolis}
\usefonttheme{professionalfonts}
\useinnertheme{circles}

% 欧文 (英語) もヒラギノ丸ゴW4に設定
\setsansfont[
  AutoFakeBold = true  % 擬似的な太字を許可
]{Hiragino Maru Gothic ProN}

% 日本語もヒラギノ丸ゴW4に設定
\setCJKsansfont[
  AutoFakeBold = true  % 擬似的な太字を許可
]{Hiragino Maru Gothic ProN}

\setbeamertemplate{navigation symbols}{}

\setbeamercolor{title}{fg=structure, bg=}
\setbeamercolor{frametitle}{fg=structure, bg=}

\newcommand{\todayAD}{\number\year 年\number\month 月\number\day 日}

\setbeamertemplate{navigation symbols}{}

\setbeamertemplate{itemize item}{\small\raise0.5pt\hbox{$\bullet$}}
\setbeamertemplate{itemize subitem}{\tiny\raise1.5pt\hbox{$\blacktriangleright$}}
\setbeamertemplate{itemize subsubitem}{\tiny\raise1.5pt\hbox{$\bigstar$}}

\newcommand{\red}[1]{\textcolor{red}{#1}}
\newcommand{\green}[1]{\textcolor{green!40!black}{#1}}
\newcommand{\blue}[1]{\textcolor{blue!80!black}{#1}}

\setbeamertemplate{frametitle}{
    \hbox to \linewidth{
        \hspace{-0.3em}
        \textcolor{darkgray}{\rule[-0.15em]{4pt}{1.0em}}
        \hspace{0.2em}
        \textbf{\insertframetitle} \hfill
    }
    \vspace{0.1em} 
}

\setbeamerfont{footnote}{size=\scriptsize}
\AtBeginSection[]{}
\setbeamerfont{frametitle}{size=\Large, series=\bfseries}
\setbeamertemplate{headline}{}

\begin{document}

%======================================================================
% Slide 1: タイトル
%======================================================================
\begin{frame}
    \centering
    \vspace{3em}
    {\LARGE \textbf{最終報告}}
    \vspace{1.5em}
    \hrule
    \vspace{1em}
    {\normalsize データに基づく救急拠点配置の検証とシミュレーション} \\
    \vspace{1em}
    \centering
    {\small G3 -安全地帯-}\\[0.5em]
    {\footnotesize
    \begin{tabular}{ccccc}
        上野 穂 & 天羽 祐太 & CHEN JUNQI & SHEN JINGJING & LI Kerun
    \end{tabular}
    }\\[1em]
    {\small \todayAD}
\end{frame}

%======================================================================
% Slide 2: 目次
%======================================================================
\begin{frame}[t]{目次}
    \begin{enumerate}
        \item 背景・課題
        \item 目的
        \item システム概要
        \item 主な機能
              \begin{itemize}
                  \item 到達圏と出動プロット
                  \item カバー率分析と比較
                  \item 拠点シミュレーション
                  \item 配置最適化
                  \item 渋滞考慮モード
                  \item アニメーション表示
              \end{itemize}
        \item 分析結果
        \item まとめ
    \end{enumerate}
\end{frame}

%======================================================================
% Slide 3: 背景・課題
%======================================================================
\section{背景・課題}
\begin{frame}[t]{背景・課題}
    \begin{itemize}
        \item \textbf{救急車増台の効果検証}
              \begin{itemize}
                  \item R4年度から救急車を増隊しているがその効果について
              \end{itemize}
        \item \textbf{拠点配置の最適化}
              \begin{itemize}
                  \item 救急需要の増加に対し更に救急車を増設するとどの地域(消防署)に増やすのが効果的か
              \end{itemize}
    \end{itemize}
    \begin{tcolorbox}[colframe=structure, colback=white, title=解決策の概要]
        各消防署からの\textbf{到達圏}を道路データに基づき可視化し、\\
        実際の\textbf{出動データ}と組み合わせて分析するシステムを開発。\\
        $\Rightarrow$ \textbf{実データに基づく意思決定}を支援する。
    \end{tcolorbox}
\end{frame}

%======================================================================
% Slide 4: 目的
%======================================================================
\section{目的}
\begin{frame}[t]{目的}
    \begin{itemize}
        \item \textbf{救急体制の可視化}
              \begin{itemize}
                  \item 松山市の各消防署から自動車で\blue{5分〜20分以内}に到達できる範囲を地図上に明示
              \end{itemize}
        \item \textbf{配置効果の検証}
              \begin{itemize}
                  \item \red{2015年(H27)}と\red{2024年(R6)}の出動データを比較
                  \item リソースの配置変更がカバー率に与えた影響を数値化
              \end{itemize}
        \item \textbf{意思決定支援}
              \begin{itemize}
                  \item 最適なリソース配置をシミュレーション可能に
              \end{itemize}
    \end{itemize}
\end{frame}

%======================================================================
% Slide 5: システム構成と使用技術
%======================================================================
\section{システム概要}
\begin{frame}[t]{システム構成と使用技術}
    \begin{itemize}
        \item \textbf{言語・フレームワーク:Python / Streamlit}
              \begin{itemize}
                  \item データ分析に特化したwebアプリケーションを作成
              \end{itemize}
        \item \textbf{地図データ基盤:OpenStreetMap}
              \begin{itemize}
                  \item \blue{実際の道路形状、一方通行、道路種別}を含む詳細データを使用
              \end{itemize}
        \item \textbf{経路探索:NetworkX}
              \begin{itemize}
                  \item 直線距離ではなく、\blue{「道のり」}での到達圏を計算
              \end{itemize}
        \item \textbf{可視化:Folium / Leaflet}
              \begin{itemize}
                  \item 拡大・縮小・クリック操作などを実現
              \end{itemize}
    \end{itemize}
\end{frame}

%======================================================================
% Slide 6: 主な機能(1) 到達圏と出動プロット
%======================================================================
\section{主な機能}
\begin{frame}[t]{機能① 到達圏と出動プロット}
    \begin{itemize}
        \item \textbf{到達圏可視化}
              \begin{itemize}
                  \item 5分 / 10分 / 15分 / 20分 ごとの到達範囲を色分け表示
              \end{itemize}
        \item \textbf{出動地点の動的表示}
              \begin{itemize}
                  \item 特定の日付を選択して出動地点を地図上にプロット可能
                  \item データから算出した時間帯別の到達圏も表示可能
              \end{itemize}
    \end{itemize}
\end{frame}

%======================================================================
% Slide 7: 機能① 到達圏表示例(スクリーンショット)
%======================================================================
\begin{frame}[t]{機能① 到達圏と出動プロット}
    %% 📸 スクショ①:5/10/15/20分圏が表示された地図画面
    %% 撮影対象:到達圏可視化ページで全消防署の到達圏を表示した状態
    \begin{figure}[H]
        \centering
        % \includegraphics[width=0.65\linewidth]{image/fig_isochrone.png}
        \fbox{\parbox{0.65\linewidth}{\centering\vspace{3em}📸 スクショ①\\到達圏表示例(5/10/15/20分圏)\vspace{3em}}}
        \caption{到達圏表示例}
    \end{figure}
\end{frame}

%======================================================================
% Slide 8: 機能② カバー率分析と比較
%======================================================================
\begin{frame}[t]{機能② カバー率分析と比較}
    \begin{itemize}
        \item \textbf{カバー率の算出}
              \begin{itemize}
                  \item 各到達圏内にどれだけの出動地点が含まれているかを自動計算
              \end{itemize}
        \item \textbf{比較モード}
              \begin{itemize}
                  \item 配置変更前(\red{H27})と変更後(\red{R6})を並べて比較
                  \item 改善度(差分)を数値で表示
              \end{itemize}
    \end{itemize}
\end{frame}

%======================================================================
% Slide 9: 機能② カバー率比較例(スクリーンショット)
%======================================================================
\begin{frame}[t]{機能② カバー率分析と比較}
    %% 📸 スクショ②:H27⇔R6の比較グラフ
    %% 撮影対象:カバー率分析ページでH27とR6を比較した画面
    \begin{figure}[H]
        \centering
        % \includegraphics[width=0.60\linewidth]{image/fig_coverage.png}
        \fbox{\parbox{0.60\linewidth}{\centering\vspace{3em}📸 スクショ②\\カバー率比較例(H27 vs R6)\vspace{3em}}}
        \caption{カバー率比較例}
    \end{figure}
\end{frame}

%======================================================================
% Slide 10: 機能③ 拠点シミュレーション
%======================================================================
\begin{frame}[t]{機能③ 拠点シミュレーション}
    \begin{itemize}
        \item \textbf{仮想消防署の追加}
              \begin{itemize}
                  \item 地図上の任意の地点に「もし消防署を置いたら」と仮定
                  \item シミュレーションによるカバー率変化を即座に確認
              \end{itemize}
        \item \textbf{活用のメリット}
              \begin{itemize}
                  \item \blue{分署の設置検討}
                  \item \blue{救急車の動態待機場所の選定}
                  \item \blue{新設によるカバー率向上効果の定量的予測}
              \end{itemize}
    \end{itemize}
\end{frame}

%======================================================================
% Slide 11: 機能③ 拠点追加例(スクリーンショット)
%======================================================================
\begin{frame}[t]{機能③ 拠点シミュレーション}
    %% 📸 スクショ③:仮想消防署追加後の到達圏変化
    %% 撮影対象:任意の地点に仮想拠点を追加し、カバー率が向上した画面
    \begin{figure}[H]
        \centering
        % \includegraphics[width=0.60\linewidth]{image/fig_simulation.png}
        \fbox{\parbox{0.60\linewidth}{\centering\vspace{3em}📸 スクショ③\\拠点シミュレーション例\vspace{3em}}}
        \caption{仮想拠点追加による到達圏変化}
    \end{figure}
\end{frame}

%======================================================================
% Slide 12: 機能④ 配置最適化
%======================================================================
\begin{frame}[t]{機能④ 配置最適化}
    \begin{itemize}
        \item \textbf{候補地点の自動生成}
              \begin{itemize}
                  \item 出動が多く、かつ既存拠点でカバーされていない地域から順に候補を選定(貪欲法)
              \end{itemize}
        \item \textbf{最適化の実行}
              \begin{itemize}
                  \item 追加する拠点数を指定して最適配置を自動計算
                  \item カバー率の改善効果を即座に確認可能
              \end{itemize}
        \item \textbf{活用シーン}
              \begin{itemize}
                  \item 限られた予算で最大の効果を得る配置の検討
              \end{itemize}
    \end{itemize}
\end{frame}

%======================================================================
% Slide 13: 機能④ 配置最適化(スクリーンショット+デモ)
%======================================================================
\begin{frame}[t]{機能④ 配置最適化}
    %% 📸 スクショ④:候補地点+改善効果の表示
    %% 🎬 動画①:候補地点を追加→カバー率が改善する様子のデモ動画
    %% 撮影対象:配置最適化ページで候補地点を選択し、カバー率が改善される過程
    \begin{figure}[H]
        \centering
        % \includegraphics[width=0.55\linewidth]{image/fig_optimization.png}
        \fbox{\parbox{0.55\linewidth}{\centering\vspace{2.5em}📸 スクショ④ / 🎬 動画①\\配置最適化結果\vspace{2.5em}}}
        \caption{最適配置候補と改善効果}
    \end{figure}
    \vspace{-0.5em}
    \centering{\small ※ パワポ変換時にデモ動画を挿入予定}
\end{frame}

\begin{frame}[t]{貪欲法による最適化}
    \begin{itemize}
        \item \textbf{アルゴリズムの概要}
              \begin{itemize}
                  \item 出動密度が高い地点を候補として抽出
                  \item 既存拠点でカバーされていない地点を優先
                  \item 1拠点ずつ追加し、都度カバー率を再計算
              \end{itemize}
        \item \textbf{特徴}
              \begin{itemize}
                  \item 計算が高速で、大規模データにも対応可能
                  \item 厳密な最適解ではないが、実用的な解を提供
              \end{itemize}
    \end{itemize}
\end{frame}

%======================================================================
% Slide 14: 機能⑤ 渋滞考慮モード
%======================================================================
\begin{frame}[t]{機能⑤ 渋滞考慮モード}
    \begin{itemize}
        \item \textbf{時間帯別の遅延係数}
              \begin{itemize}
                  \item 実データから時間帯ごとの到達時間の傾向を学習
                  \item 深夜帯とラッシュ時で到達圏が変化
              \end{itemize}
        \item \textbf{曜日別の補正}
              \begin{itemize}
                  \item 平日・休日で異なる交通状況を反映
              \end{itemize}
        \item \textbf{より現実的な分析が可能に}
              \begin{itemize}
                  \item 「朝のラッシュ時にどこまで到達できるか」を可視化
              \end{itemize}
    \end{itemize}
\end{frame}

%======================================================================
% Slide 15: 機能⑤ 渋滞考慮モード(時間帯別スクリーンショット)
%======================================================================
\begin{frame}[t]{機能⑤ 渋滞考慮モード ― 時間帯別}
    %% 📸 スクショ⑤:深夜帯の到達圏
    %% 📸 スクショ⑥:朝ラッシュ時の到達圏(比較用)
    %% 撮影対象:渋滞考慮モードで深夜とラッシュ時を比較した画面
    \begin{figure}[H]
        \centering
        % \subfloat[深夜帯]{\includegraphics[width=0.45\linewidth]{image/fig_night.png}}\hspace{0.5em}
        % \subfloat[ラッシュ時]{\includegraphics[width=0.45\linewidth]{image/fig_rush.png}}
        \fbox{\parbox{0.42\linewidth}{\centering\vspace{2em}📸 スクショ⑤\\深夜帯\vspace{2em}}}
        \hspace{0.5em}
        \fbox{\parbox{0.42\linewidth}{\centering\vspace{2em}📸 スクショ⑥\\ラッシュ時\vspace{2em}}}
        \caption{時間帯による到達圏の違い}
    \end{figure}
\end{frame}

%======================================================================
% Slide 16: 機能⑤ 渋滞考慮モード(曜日別スクリーンショット)
%======================================================================
\begin{frame}[t]{機能⑤ 渋滞考慮モード ― 曜日別}
    %% 📸 スクショ⑦:平日の到達圏
    %% 📸 スクショ⑧:休日の到達圏(比較用)
    %% 撮影対象:渋滞考慮モードで平日と休日を比較した画面
    \begin{figure}[H]
        \centering
        % \subfloat[平日]{\includegraphics[width=0.45\linewidth]{image/fig_weekday.png}}\hspace{0.5em}
        % \subfloat[休日]{\includegraphics[width=0.45\linewidth]{image/fig_weekend.png}}
        \fbox{\parbox{0.42\linewidth}{\centering\vspace{2em}📸 スクショ⑦\\平日\vspace{2em}}}
        \hspace{0.5em}
        \fbox{\parbox{0.42\linewidth}{\centering\vspace{2em}📸 スクショ⑧\\休日\vspace{2em}}}
        \caption{曜日による到達圏の違い}
    \end{figure}
\end{frame}

%======================================================================
% Slide 17: 機能⑥ アニメーション表示
%======================================================================
\begin{frame}[t]{機能⑥ アニメーション表示}
    \begin{itemize}
        \item \textbf{1日の出動を時系列で再生}
              \begin{itemize}
                  \item 出動地点が時間経過とともに地図上に出現
                  \item 出動の「流れ」を直感的に把握可能
              \end{itemize}
        \item \textbf{活用シーン}
              \begin{itemize}
                  \item 特定日の出動パターンの振り返り
                  \item 時間帯別の出動集中エリアの特定
                  \item プレゼンテーション・説明資料での活用
              \end{itemize}
    \end{itemize}
\end{frame}

%======================================================================
% Slide 17: 機能⑥ アニメーション表示(デモ動画)
%======================================================================
\begin{frame}[t]{機能⑥ アニメーション表示}
    %% 🎬 動画②:1日の出動が時系列で再生される様子(メインデモ)
    %% 撮影対象:アニメーションページで1日分の出動を再生
    \begin{figure}[H]
        \centering
        % \includegraphics[width=0.60\linewidth]{image/fig_animation.png}
        \fbox{\parbox{0.60\linewidth}{\centering\vspace{3em}🎬 動画②\\アニメーションデモ\vspace{3em}}}
        \caption{出動の時系列再生}
    \end{figure}
    \centering{\small ※ パワポ変換時にデモ動画を挿入予定}
\end{frame}

%======================================================================
% Slide 18: 分析結果・成果
%======================================================================
\section{分析結果}
\begin{frame}[t]{分析結果・成果}
    \begin{itemize}
        \item \textbf{カバー率の改善を確認}
              \begin{itemize}
                  \item H27 → R6 で○○\%のカバー率向上を数値化
                        %% TODO: 具体的な数値を入れる
              \end{itemize}
        \item \textbf{最適配置シミュレーション}
              \begin{itemize}
                  \item 特定エリアへの拠点追加で○○\%の改善効果
                        %% TODO: 具体的な数値を入れる
              \end{itemize}
        \item \textbf{時間帯別の差異を可視化}
              \begin{itemize}
                  \item ラッシュ時は深夜帯と比較して到達圏が○○\%縮小
                        %% TODO: 具体的な数値を入れる
              \end{itemize}
    \end{itemize}
    \vspace{0.5em}
    \centering{\small ※ 具体的な数値は最終データ集計後に記入}
\end{frame}

%======================================================================
% Slide 19: まとめ
%======================================================================
\section{まとめ}
\begin{frame}[t]{まとめ}
    \begin{itemize}
        \item \textbf{達成した目標}
              \begin{itemize}
                  \item 救急体制の可視化システムを構築
                  \item 過去データに基づく配置効果の検証が可能に
                  \item 仮想拠点シミュレーション・最適化機能を実装
              \end{itemize}
        \item \textbf{今後の展望}
              \begin{itemize}
                  \item リアルタイム交通情報との連携
                  \item 複数拠点の同時最適化アルゴリズムの改良
                  \item 他自治体への展開可能性の検討
              \end{itemize}
    \end{itemize}
\end{frame}

%======================================================================
% Appendix(目次に含まない)
%======================================================================
\appendix

\begin{frame}[t]{速度係数の算出}
    \begin{itemize}
        \item 使用データ:令和6年松山市救急出動データ
        \item 各出動記録について、次の式で速度係数を算出
              \[
                  \text{速度係数} = \frac{\text{{現場到着時間(min)}}}{\text{距離 (km)}}
              \]
        \item 全出動記録の速度係数の平均を算出
        \item 得られた平均値を用いて、各到達圏の距離を計算
    \end{itemize}
\end{frame}

\end{document}

%======================================================================
% 撮影チェックリスト(全8点)
%======================================================================
% 
% 【スクリーンショット(6点)】
% □ スクショ①:到達圏可視化(5/10/15/20分圏が表示された地図)
%     → 対象画面:到達圏可視化ページで全消防署の到達圏を表示
% 
% □ スクショ②:カバー率比較(H27⇔R6の比較グラフ)
%     → 対象画面:カバー率分析ページでH27とR6を比較
% 
% □ スクショ③:拠点シミュレーション(仮想消防署追加後の到達圏変化)
%     → 対象画面:任意の地点に仮想拠点を追加し、カバー率が向上した画面
% 
% □ スクショ④:配置最適化結果(候補地点+改善効果の表示)
%     → 対象画面:配置最適化ページで候補地点が表示された画面
% 
% □ スクショ⑤:渋滞考慮モード・深夜帯の到達圏
%     → 対象画面:渋滞考慮モードで深夜(例:2:00)を選択
% 
% □ スクショ⑥:渋滞考慮モード・ラッシュ時の到達圏
%     → 対象画面:渋滞考慮モードでラッシュ時(例:8:00)を選択
% 
% 【動画(2点)】
% □ 動画①:配置最適化デモ(20〜30秒程度)
%     → 内容:候補地点を選択→カバー率が改善する様子
%     → 操作:配置最適化ページで拠点数を指定→最適化実行→結果確認
% 
% □ 動画②:アニメーションデモ(30〜60秒程度)
%     → 内容:1日の出動が時系列で再生される様子
%     → 操作:アニメーションページで日付選択→再生ボタン→時系列で出動が表示
% 
%======================================================================