% \documentclass[dvipdfmx, 11pt]{beamer}
\documentclass[aspectratio=169, dvipdfmx, 14pt]{beamer} % aspectratio=43, 149, 169
\usepackage{here, amsmath, latexsym, amssymb, bm, ascmac, mathtools, multicol, tcolorbox, subfig, url}
\usepackage{graphicx}
\usepackage{multimedia}  % 動画埋め込み用
\usepackage{tikz}        % 図形描画用
\usepackage{pgfpages}
\usepackage{ascmac}
\usepackage{bm}
\usepackage{atbegshi} %しおりの文字化け解消
\usepackage{xcolor}
\usepackage{hyperref}
\usepackage{appendixnumberbeamer}
\pdfstringdefDisableCommands{\def\translate#1{#1}}

\newcounter{mainframenumber}
\setcounter{mainframenumber}{0}

\usetheme{metropolis}
\usefonttheme{professionalfonts}
\useinnertheme{circles}
\useoutertheme{infolines}

\ifnum 42146=\euc"A4A2
\AtBeginShipoutFirst{\special{pdf:tounicode EUC-UCS2}}
\else
\AtBeginShipoutFirst{\special{pdf:tounicode 90ms-RKSJ-UCS2}}
\fi

\setbeamertemplate{navigation symbols}{}
\renewcommand{\kanjifamilydefault}{\gtdefault}

\setbeamercolor{title}{fg=structure, bg=}
\setbeamercolor{frametitle}{fg=structure, bg=}

\newcommand{\todayAD}{\number\year 年\number\month 月\number\day 日}

\setbeamertemplate{navigation symbols}{}


\setbeamertemplate{itemize item}{\small\raise0.5pt\hbox{$\bullet$}}
\setbeamertemplate{itemize subitem}{\tiny\raise1.5pt\hbox{$\blacktriangleright$}}
\setbeamertemplate{itemize subsubitem}{\tiny\raise1.5pt\hbox{$\bigstar$}}

\newcommand{\red}[1]{\textcolor{red}{#1}}
\newcommand{\green}[1]{\textcolor{green!40!black}{#1}}
\newcommand{\blue}[1]{\textcolor{blue!80!black}{#1}}

\setbeamertemplate{frametitle}{
    \hbox to \linewidth{
        \hspace{-0.3em}
        \textcolor{darkgray}{\rule[-0.15em]{4pt}{1.0em}}
        \hspace{0.2em}
        \textbf{\insertframetitle} \hfill
    }
    \vspace{0.1em} 
}

\setbeamerfont{footnote}{size=\scriptsize}
\AtBeginSection[]{}
\setbeamerfont{frametitle}{size=\Large, series=\bfseries}
\setbeamertemplate{headline}{}

\begin{document}

%======================================================================
% Slide 1: タイトル
%======================================================================
\begin{frame}
    \centering
    \vspace{3em}
    {\LARGE \textbf{中間報告}}
    \vspace{1.5em}
    \hrule
    \vspace{1em}
    {\normalsize データに基づく救急拠点配置の検証とシミュレーション}
    \vspace{1em}
    \centering
    {\small G3 -安全地帯-}\\[0.5em]
    {\footnotesize
    \begin{tabular}{ccccc}
        上野 穂 & 天羽 祐太 & CHEN JUNQI & SHEN JINGJING & LI Kerun
    \end{tabular}
    }\\[1em]
    {\small \todayAD}
\end{frame}

%======================================================================
% Slide 2: 目次
%======================================================================
\begin{frame}[t]{目次}
    \tableofcontents
\end{frame}

%======================================================================
% Slide 3: 課題
%======================================================================
\section{課題}
\begin{frame}[t]{課題}
    \begin{itemize}
        \item \textbf{救急車増台の効果検証}
              \begin{itemize}
                  \item R4年度から救急車を増隊しているがその効果について
              \end{itemize}
        \item \textbf{拠点配置の最適化}
              \begin{itemize}
                  \item 救急需要の増加に対し更に救急車を造替するとどの地域(消防署)に増やすのが効果的か
              \end{itemize}
    \end{itemize}
    \begin{tcolorbox}[colframe=structure, colback=white, title=解決策の概要]
        各消防署からの\textbf{到達圏}を道路データに基づき可視化し、\\
        実際の\textbf{出動データ}と組み合わせて分析するシステムを開発。\\
        $\Rightarrow$ \textbf{実データに基づく意思決定}を支援する。
    \end{tcolorbox}
\end{frame}

%======================================================================
% Slide 3: 目的
%======================================================================
\section{目的}
\begin{frame}[t]{目的}
    \begin{itemize}
        \item \textbf{救急体制の可視化}
              \begin{itemize}
                  \item 松山市の各消防署から自動車で\blue{5分〜20分以内}に到達できる範囲を地図上に明示
              \end{itemize}
        \item \textbf{配置効果の検証}
              \begin{itemize}
                  \item \red{2015年(H27)}と\red{2024年(R6)}の出動データを比較
                  \item リソースの配置変更がカバー率に与えた影響を数値化
              \end{itemize}
        \item \textbf{意思決定支援}
              \begin{itemize}
                  \item 最適なリソース配置をシミュレーション可能に
              \end{itemize}
    \end{itemize}
\end{frame}

%======================================================================
% Slide 4: システム構成と使用技術
%======================================================================
\section{システム構成}
\begin{frame}[t]{システム構成と使用技術}
    \begin{itemize}
        \item \textbf{言語・フレームワーク:Python / Streamlit}
              \begin{itemize}
                  \item データ分析に特化したwebアプリケーションを作成
              \end{itemize}
        \item \textbf{地図データ基盤:OpenStreetMap}
              \begin{itemize}
                  \item \blue{実際の道路形状、一方通行、道路種別}を含む詳細データを使用
              \end{itemize}
        \item \textbf{経路探索:NetworkX}
              \begin{itemize}
                  \item 直線距離ではなく、\blue{「道のり」}での到達圏を計算
              \end{itemize}
        \item \textbf{可視化:Folium / Leaflet}
              \begin{itemize}
                  \item 拡大・縮小・クリック操作などを実現
              \end{itemize}
    \end{itemize}
\end{frame}

%======================================================================
% Slide 5: 主な機能(1) 到達圏と出動プロット
%======================================================================
\section{主な機能}
\begin{frame}[t]{主な機能(1) 到達圏と出動プロット}
    \begin{itemize}
        \item \textbf{到達圏可視化}
              \begin{itemize}
                  \item 5分 / 10分 / 15分 / 20分 ごとの到達範囲を色分け表示
              \end{itemize}
        \item \textbf{出動地点の動的表示}
              \begin{itemize}
                  \item 特定の日付を選択して出動地点を地図上にプロット可能
                  \item データから算出した時間帯別の到達圏も表示可能 \\
                        (係数を調整して到達圏を拡大・縮小)
              \end{itemize}
    \end{itemize}
\end{frame}

%======================================================================
% Slide 6: 主な機能(1) 到達圏表示例
%======================================================================
\begin{frame}[t]{主な機能(1) 到達圏と出動プロット}
    \begin{figure}[H]
        \centering
        \subfloat[到達圏表示例]{\includegraphics[width=0.60\linewidth]{image/fig:map.png}}\hspace{1em}
    \end{figure}
\end{frame}

%======================================================================
% Slide 7: 主な機能(1) 日別比較例
%======================================================================
\begin{frame}[t]{主な機能(1) 到達圏と出動プロット}
    \begin{figure}[H]
        \centering
        \subfloat[到達圏表示例]{\includegraphics[width=0.70\linewidth]{image/fig:compareDay.png}}\hspace{1em}
    \end{figure}
\end{frame}

%======================================================================
% Slide 8: 主な機能(2) カバー率分析と比較
%======================================================================
\begin{frame}[t]{主な機能(2) カバー率分析と比較}
    \begin{itemize}
        \item \textbf{カバー率の算出}
              \begin{itemize}
                  \item 各到達圏内にどれだけの出動地点が含まれているかを自動計算
              \end{itemize}
        \item \textbf{比較モード}
              \begin{itemize}
                  \item 配置変更前(\red{H27})と変更後(\red{R6})を並べて比較
                  \item 改善度(差分)を数値で表示
                  \item ただし、単純に道路情報のみを考慮したものなので、実際に変化したリソース数を踏まえた評価は今後の課題
              \end{itemize}
    \end{itemize}
\end{frame}

%======================================================================
% Slide 9: 主な機能(2) カバー率比較例
%======================================================================
\begin{frame}[t]{主な機能(2) カバー率分析と比較}
    \begin{figure}[H]
        \centering
        \subfloat[カバー率比較例]{\includegraphics[width=0.60\linewidth]{image/fig:cover.png}}\hspace{1em}
    \end{figure}
\end{frame}

%======================================================================
% Slide 10: 主な機能(3) 拠点シミュレーション
%======================================================================
\begin{frame}[t]{主な機能(3) 拠点シミュレーション}
    \begin{itemize}
        \item \textbf{仮想消防署の追加}
              \begin{itemize}
                  \item 地図上の任意の地点に「もし消防署を置いたら」と仮定
                  \item シミュレーションによるカバー率変化を即座に確認
              \end{itemize}
        \item \textbf{活用のメリット}
              \begin{itemize}
                  \item \blue{分署の設置検討}
                  \item \blue{救急車の動態待機場所の選定}
                  \item \blue{新設によるカバー率向上効果の定量的予測}
              \end{itemize}
    \end{itemize}
    \vspace{0.1em}
\end{frame}

%======================================================================
% Slide 11: 主な機能(3) 拠点追加例
%======================================================================
\begin{frame}[t]{主な機能(3) 拠点シミュレーション}
    \begin{figure}[H]
        \centering
        \subfloat[拠点追加例]{\includegraphics[width=0.60\linewidth]{image/fig:addbase.png}}\hspace{1em}
    \end{figure}
\end{frame}

%======================================================================
% Slide 12: 主な機能(3) 愛大拠点シミュレーション例
%======================================================================
\begin{frame}[t]{主な機能(3) 拠点シミュレーション}
    \begin{figure}[H]
        \centering
        \subfloat[愛大拠点シミュレーション例]{\includegraphics[width=0.50\linewidth]{image/fig:aidaibase.png}}\hspace{1em}
    \end{figure}
\end{frame}

%======================================================================
% Slide 13: 今後の展望
%======================================================================
\section{今後の展望}
\begin{frame}[t]{今後の展望}
    \begin{itemize}
        \item \textbf{精度の向上}
              \begin{itemize}
                  \item 時間帯別の交通量変化(ラッシュ時など)を考慮した分析の検討
                  \item リアルタイム交通情報の活用可能性の調査
              \end{itemize}
        \item \textbf{シミュレーション機能の強化}
              \begin{itemize}
                  \item 現時点での仮想拠点は手動配置のみ
                  \item 最適配置候補を自動提案するアルゴリズムの検討
              \end{itemize}
        \item \textbf{到達圏分析の多様化}
              \begin{itemize}
                  \item 各消防署の保有リソース(救急車台数など)を考慮した到達圏分析手法の検討
              \end{itemize}
    \end{itemize}
\end{frame}

%======================================================================
% Appendix(目次に含まない)
%======================================================================
\appendix

\begin{frame}[t]{速度係数の算出}
    \begin{itemize}
        \item 使用データ:令和6年松山市救急出動データ
        \item 各出動記録について、次の式で速度係数を算出
              \[
                  \text{速度係数} = \frac{\text{{現場到着時間(min)}}}{\text{距離 (km)}}
              \]
        \item 全出動記録の速度係数の平均を算出
        \item 得られた平均値を用いて、各到達圏の距離を計算
    \end{itemize}
\end{frame}

\end{document}